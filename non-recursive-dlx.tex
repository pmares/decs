\documentclass[a4paper]{article}

% Import packages
\usepackage{amsmath, amsfonts, amsthm}  % AMS math libraries
\usepackage{amssymb}       % Large extended set of symbols: http://home.online.no/~pjacklam/latex/textcomp.pdf
\usepackage{multirow}      % Row spanning in tables.
\usepackage{graphicx}      % Extended graphics library
%\usepackage{textcomp}      % Extended set of symbols: http://home.online.no/~pjacklam/latex/textcomp.pdf
%\usepackage{varioref}      % Use \vref instead of \ref
%\usepackage{wrapfig}
%\usepackage{subfigure}
%\usepackage{url}
\usepackage[pdftex,
	pdfauthor={Jan Magne Tjensvold},
	pdftitle={Non-Recursive Dancing Links},
	pdfkeywords={Iterative; Recursive; Dancing Links; DLX},
	pdfsubject={Algorithms},
	pdflang={en},
	bookmarks,
	bookmarksnumbered,
]{hyperref}  % Sets pdfTeX to include bookmarks in the output file (Should always be the last package included)
\usepackage{algorithm}     % Pseudocode float environment
\usepackage{algorithmic}   % Pseudocode
\usepackage[noend, nobraces]{javadistalgo}  % Java style pseudocode

\graphicspath{{./}{images/}}

% Limit the depth of the table of contents
%\setcounter{tocdepth}{1}

% Custom commands

%\renewcommand{\bmod}{\mbox{ mod }}

%\makeatletter
%\def\imod#1{\allowbreak\mkern10mu({\operator@font mod}\,\,#1)}
%\makeatother

%\DeclareMathOperator{\lcm}{lcm}

% Document info
\title{Non-Recursive Dancing Links}
\author{Jan Magne Tjensvold}
\date{\today}



% Here the actual document begins.
\begin{document}

\maketitle  % Report titlepage

% Include external file
%
\chapter*{Abstract}
\addcontentsline{toc}{chapter}{Abstract}


  % Abstract

% Table of contents, tables and figures
%\tableofcontents
%\listoftables
%\listoffigures

In \cite{knuth00dancing} Donald Knuth explains a backtracking algorithm he calls Dancing Links.
As a result of being a backtracking algorithm it is recursive as shown in Algorithm \ref{alg:search}.
To turn it into a non-recursive algorithm the recursive call can be eliminated by adding \textbf{goto} statements and using a stack.
The resulting Algorithm \ref{alg:nrsearch} is a non-recursive Dancing Links algorithm.
In this algorithm each of the \textbf{foreach} loops has been replaced with equivalent \textbf{while} loops.

The use of goto statements is generally frowned upon because they have a tendency to make the code very hard to understand, and harder still to modify and debug.
To limit the number of goto statements the recursive base case has been moved inside the while loop.
Another goto statement has been eliminated by using the \textbf{continue} keyword, present in many modern programming languages, to jump to the beginning of the while loop.
The continue keyword is basically a limited goto statement and it is considered more acceptable in use than goto.

That leaves one goto statement that I have been unable to eliminate.
Unfortunately it is quite disruptive as it jumps into the while loop which might result in variables not being properly initialized.
However, this should not be a problem if all the variables have been declared outside the while loop.
Algorithm \ref{alg:nrsearch} has eliminated the recursive call, but given the remaining goto it is more appropriate to call it non-recursive than iterative.
Call it a hybrid if you will, or maybe Frankenstein monster is more appropriate.

Testing and validating the correctness of the modified algorithm is left as an exercise for the reader.
Can you find a way to eliminate the last goto statement as well?


\begin{algorithm}[p]
	\caption{Dancing Links recursive search.}
	\label{alg:search}
	\begin{distribalgo}[1]
		\PROCEDURE{search($k$)}
			\IF{$h.right = h$}
				\STATE Print solution and return.  \COMMENT{Base case for the recursion}
			\ENDIF
			\STATE $c \leftarrow$ choose\_column()
			\STATE cover($c$)
			\FOREACH{$r \leftarrow c.down, c.down.down, \ldots,$ \textbf{while} $r \neq c$}
				\STATE $O_k \leftarrow r$  \COMMENT{Add $r$ to partial solution}
				\FOREACH{$j \leftarrow r.right, r.right.right, \ldots,$ \textbf{while} $j \neq r$}
					\STATE cover($j.column$)
				\ENDFOR
				\STATE search($k + 1$)
				\FOREACH{$j \leftarrow r.left, r.left.left, \ldots,$ \textbf{while} $j \neq r$}
					\STATE uncover($j.column$)
				\ENDFOR
			\ENDFOR
			\STATE uncover($c$)
		\ENDPROC
	\end{distribalgo}
\end{algorithm}


\begin{algorithm}[p]
	\caption{Dancing Links non-recursive search.}
	\label{alg:nrsearch}
	\begin{distribalgo}[1]
		\PROCEDURE{search()}
			\STATE $k \leftarrow 0$
			\STATE $S \leftarrow$ new stack
			\STATE $c \leftarrow$ choose\_column()
			\STATE cover($c$)
			\STATE $r \leftarrow c.down$
			\WHILE{$r \neq c$}
				\STATE $O_k \leftarrow r$  \COMMENT{Add $r$ to partial solution}
				\STATE $j \leftarrow r.right$
				\WHILE{$j \neq r$}
					\STATE cover($j.column$)
					\STATE $j \leftarrow j.right$
				\ENDWHILE
				\IF{$h.right = h$}
					\STATE Print solution.  \COMMENT{Former recursion base case}
				\ELSE
					\STATE $k \leftarrow k + 1$
					\STATE $S.push(r)$
					\STATE $c \leftarrow$ choose\_column()
					\STATE cover($c$)
					\STATE $r \leftarrow c.down$
					\STATE \textbf{continue}  \COMMENT{Go to the beginning of the while loop}
					\STATE kitchen\_sink:  \COMMENT{Here we removed the recursive call}
					\STATE $r \leftarrow S.pop()$
					\STATE $c \leftarrow r.column$
					\STATE $k \leftarrow k - 1$
				\ENDIF
				\STATE $j \leftarrow r.left$
				\WHILE{$j \neq r$}
					\STATE uncover($j.column$)
					\STATE $j \leftarrow j.left$
				\ENDWHILE
				\STATE $r \leftarrow r.down$
			\ENDWHILE
			\STATE uncover($c$)
			\IF{$k > 0$}
				\STATE \textbf{goto} kitchen\_sink  \COMMENT{Here there be monsters}
			\ENDIF
		\ENDPROC
	\end{distribalgo}
\end{algorithm}



% BibTeX compiles the bibliography
%\phantomsection
%\addcontentsline{toc}{section}{References}
\bibliographystyle{custom-r2}  % custom bibliography style
\bibliography{local}  % import local.bib

\end{document}

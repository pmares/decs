
\chapter{Conclusion}
\label{conclusion}

% Description of important/interesting implementation details.



\section{Future work}

\subsection*{Implement a compression scheme}

To ease the file format explanation and implementation no compression technique was implemented.
In cases where an exact cover problem has a very large number of solutions it would be an advantage to have some sort of compression.
A technique called bit packing described in \cite{gdip_bitpack, gpg_bitpack} provides a way to do some simple compression with a very low performance overhead.
By applying this technique we can exploit the otherwise unused bits.
The bit packing scheme can also be generalized to pack the data even more densely by doing sub-bit precision packing.
The trade off between processing overhead and file size can be beneficial in cases where the number of solutions is large and where storage and bandwidth resources are scarce.
It might also be worth looking into other compression schemes which are easy to implement.


\subsection*{Implement more transforms}

Currently only the $n$-queens transform has been implemented.
The other transforms explained in the report could be implemented as well making DECS more useful.


\subsection*{Improve the Petri net simulation model}

The Petri net model could be made more complete by incorporating other aspects of distributed computing as well.
Simulating client failure or malicious clients submitting incorrect data could be a possible extension.
That would require a certain piece of the problem to be sent to multiple clients for redundancy and verification.
It would also be a good idea to try to eliminate some of assumptions currently present in the simulation.
This would make the simulation results more accurate and true to the actual system.




%\subsection*{Implement a matrix hashing function}

%To 
% TODO Describe caching idea.
% Equivalence of matrixes with same solutions. Translations give same matrix, etc.
